\documentclass[10pt,a5paper]{book}
\usepackage[bookmarks]{hyperref}
\usepackage[outer=1.0cm,inner=1.8cm,top=1.8cm,bottom=1.5cm]{geometry}
\usepackage[portuguese]{babel}
\usepackage[chorded,onesongcolumn]{songs}

% Configuração do cabeçalho e rodapé com fancyhdr
\usepackage{fancyhdr}
\fancyhf{}
\fancyhead[LE,RO]{\sffamily\small Songbook Legião Urbana - Real Sigma Music} % Cabeçalho dinâmico
\fancyfoot[C]{\sffamily\footnotesize Samuel Santana - Rio Real - BA} % Número de página centralizado no rodapé
\fancyfoot[LE]{\sffamily\thepage}
\fancyfoot[RO]{\sffamily\thepage}
\renewcommand{\headrulewidth}{1pt} % Linha horizontal no cabeçalho
\renewcommand{\footrulewidth}{1pt} % Linha horizontal no rodapé
\pagestyle{fancy}

% Configuração do hyperref (links sem cores e sem bordas)
\hypersetup{
	colorlinks=false,
	pdfborder={0 0 0}
}

% Personalização dos índices do pacote songs
\renewcommand{\idxheadfont}{\sffamily\bfseries\LARGE} % Cabeçalho
\renewcommand{\idxtitlefont}{\sffamily\normalsize} % Título das músicas

% Personalização do layout do songbook
\setlength{\songnumwidth}{1cm} % Largura da caixa de número da música
\renewcommand{\stitlefont}{\sffamily\bfseries\LARGE} % Fonte do título da música
\renewcommand{\extendprelude}{
	\sffamily\small % Tamanho da fonte do prelude
	\showrefs % Exibe referências bíblicas
	\showauthors % Exibe autores
	{\sffamily\songcopyright\par}
}
\renewcommand{\makepostlude}{\resettitles} % Limpa títulos ao final da música

% Configuração das fontes e espaçamentos para acordes e letras
\renewcommand{\printchord}[1]{\sffamily\bfseries\normalsize#1} % Fonte dos acordes
\renewcommand{\lyricfont}{\sffamily\normalsize} % Fonte da letra da música
\renewcommand{\clineparams}{\baselineskip=10pt} % Espaço entre acorde e letra
\renewcommand{\sharpsymbol}{\#} % sustenido
\renewcommand{\flatsymbol}{b} % bemol

% Outras configurações do pacote songs
\setlength{\cbarwidth}{1pt}  % Largura da barra vertical (refrão)
\setlength{\sbarheight}{0pt} % Altura da barra horizontal entre as músicas
\newindex{index}{index} % Definição do índice para as músicas
\minfrets=5 % Número mínimo de trastes no \gtab{}